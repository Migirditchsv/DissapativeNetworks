\documentclass{paper}
\usepackage[onehalfspacing]{setspace}
\usepackage[a4paper, total={6.25in, 9in}]{geometry}
\usepackage{amsmath}
\usepackage[numbers]{natbib}

\title{Stability and Flexibility in Dissipative Networks \\ \textnormal{Spring 2019 Network Science Project}}
\author{S. Migirditch, TF. Varley}

\begin{document}
	
	\maketitle
	\section*{Abstract}
	Many systems in nature can be plausibly modeled as dissipative structures, where complex structures emerge from the flow of energy through a far-from-equilibrium system. Many of these structures, such as ecosystems, biological systems, and systems of interacting molecules are commonly modeled as networks, where the striking complexity emerges from the interactions between many interacting components, described as edges and nodes respectively. In this project, we intend to explore general principles of how such dissipated structures behave, and the topological qualities of the associated networks, by generating a series of networks using model dissipative systems. In the best traditions of complex systems science, we hope that by abstracting the key elements of dissipative processes, our results will generalize to various areas described by this model, such as ecology and economics. The primary questions we are interested in are: given various parameters that control the flow of resources through the network, what network topologies begin to dominate, and, if the amount of energy flowing into the system is constricted, how effectively can the system adapt to maintain it's previous level of complexity? 
	\newline
	\section{Introduction \& Prior Work}
	At a cursory glance, complex systems seem to defy the inescapable truth of the 2nd Law of Thermodynamics, that low-entropy, ordered states are inevitably replaced by high-entropy disordered states. The spectacular emergence of complex life, from individuals to ecosystems seems to fly in the face of this law (a fact that proponents of intelligent design have repeatedly, if erroneously, pounced on to defend their position\footnote{
		For an interesting discussion of this, and why creationists are wrong, see \cite{thwaites_biological_2008}}).  
	There is, of course, no actual conflict between the existence of complex systems in nature and the 2nd Law of Thermodynamics: complexity can emerge as part of what is known as a dissipative structure (DS), where in far-from-equilibrium systems, a local decrease in entropy can occur if this drives the overall system towards equilibrium faster than it otherwise would. In his book \textit{Why Information Grows: The Evolution of Order, from Atoms to Economies}, Cesar Hildago provides an intuitive example of a dissipative system known to almost anyone: the humble bathtub whirlpool \cite{hidalgo_why_2015}. A full bathtub is a simple far-from-equilibrium system: the water in the bath has potential energy and "wants" to flow down the drain and follow gravity to it's equilibrium point. When the drain is opened, the formation of the relatively low-entropy whirlpool allows the water to flow down with maximal efficiency (as the whirlpool forces a sort of laminar flow which is much more efficient than turbulent flow). So, a local decrease in entropy allows the system to achieve equilibrium faster than it would otherwise and that decrease in entropy is sustained only a long as there is sufficient flow through it. 
	
	While a bathtub whirlpool is not something that one would typically model with a network, other systems commonly thought of as dissipative are also frequently described as networks. Quite a bit of ink has been spilled describing how dissipative structures may be a plausible mechanism by which life may have emerged \cite{anderson_broken_1987,shvartsev_self-organizing_2009,serafino_abiogenesis_2016}, metabolism is driven in individual organisms \cite{toussaint_thermodynamics_1998,karsenti_self-organization_2008,de_la_fuente_metabolic_2014}, ecosystems develop and self-regulate \cite{segel_dissipative_1972,kay_nonequilibrium_1991} and economics \cite{chen_unity_2016}. In all of these systems, a thermodynamically unfavourable level of organization and complexity is maintained through efficiently dissipating energy faster than it would otherwise. 
	
	To model this process, we propose a simple, generative network model, that begins with a single, high-value resource node that is slowly drained by an increasing chorus of nodes that attach to either the central node, or each-other. Each node will have an input/output function and metabolic requirements, dictating how much they must take in to survive, how much they produce, and a range of possible other nodes they can attach to. Each node will have an individual goal of satisfying it's own energy needs while the overall network will maximize the flux of energy through it's branches. 
	
	We hypothesize that such a dissipative network will take on a fractal topology, as fractal structures are thought to optimize both the efficiency with which resources can be dissipated, and the efficiency with which the system can adapt \cite{seely_fractal_2012}. Networks with fractal topology have been found exist in a range of natural contexts, from protein-protein interaction networks to the Internet \cite{song_self-similarity_2005,song_how_2007} and human functional connectivity networks \cite{gallos_small_2012}. 
	
	Finally, our findings would have implications relevant to a serious issue facing humanity: climate change, and the imperative to quickly, and permanently reduce the carbon dioxide (CO$_{2}$) and other waste-products into the environment. At the risk of veering into unwarranted speculation, the emergence of complex, industrial society could be considered a type of dissipative structure in-and-of itself. At the beginning of the 18th century, the large amounts of energy stored under the surface of the Earth represented a very far-from-equilibrium system: chemical potential energy trapped in the form of carbon-carbon bonds. When burned for fuel, the energy released can be used for work, and the oxidized products of the combustion reaction fly off to entropic heaven, resulting in an overall decrease in potential energy and significant increase in entropy. Like all dissipative structures, the burning of fossil fuels has allowed us to build ever more complex technologies and infrastructure that, in turn require us to extract and burn more hydrocarbons. The process in some senses looks quite a bit like Hildago's bathtub\cite{hidalgo_why_2015}\footnote{
		We realize that this is largely an argument-from-analogy, which, while not the strongest form of inductive reasoning, is an under-appreciated element of all scientific investigation.}:
	a far-from-equilibrium system rushes towards equilibrium, and, along the way, creates a locally complex system to make the run to equilibrium as fast as possible. 
	
	If this is a meaningful description of the last several centuries of technological development, then it raises some pertinent questions about the feasibility of various alternatives to fossil fuels currently being discussed by scientists, engineers, and politicians in response to the growing climate crisis. Questions which, we hope our model might be able to shed some light on. For instance, one question we hope to explore is that of redundancy in our dissipative network: how well can the network maintain it's internal complexity and a target number of leaf nodes when the amount of energy flowing through it is constricted. If it turns out that there is significant redundancy and the network can support it's high level leaf nodes well with fewer input resources by trimming fat, then that's a good sign that perhaps sustainability-through-efficiency is a viable method to reduce the release of CO$_{2}$: lower energy technologies like wind and solar might be able to sustain a current standard of living through creative improvements in efficiency. On the other hand, if even small decreases in the energetic input to our network cause it to collapse, that may suggest that reducing the amount of energy our world uses is not a viable strategy to maintain 21st century standards of living, and instead other, non-fossil-fuel based sources of energy (such as nuclear fission) are a necessity. 
	
	\section{Methods}
	
	\subsection{The Generative Model}
	
	Two classes of experiments must be carried out, one in which global dissipation is optimized and another in which each node attempts to maximize it's own dissipation. Different methods exist for both local and global optimization, the challenge will be to study the structures generated vary by global versus greedy optimization algorithms and not the specifics of the algorithms themselves. We will attempt to mitigate this problem on two fronts, by attempting to use the most general optimization problem possible, while also exploring structures that persist across different algorithms. 
	
	At initialization a rescource node is first created with a volume $V_R$ and a scalar value $n_R=0$ the niche score. For each time step up to $t_{seed}$ time steps after $t=0$ a producer node is added to the network 
	
	Global optimization algorithms are theoretically the easiest to generalize, as there is an adjacency matrix for each time step which optimizes diffusion. This optimal game may become computationally intractable for modestly sized networks. We suspect that partial sampling and heuristic based global optimization schemes will produce reasonable near optimal networks. Additionally near optimal networks are more relevant to applications as there are few examples of perfectly globally optimized networks in nature.
	
	Locally optimized networks face a greater risk of falling into specificity, ***
	
	\subsection{Analysis}
	Since it is unclear exactly what kind of structure will be generated by this model, and how different parameters will effect the resulting network topology, our first priority will be to test various parameters and do holistic analysis of the resulting networks: characterizing qualities such as the degree distribution, average path length, etc. As mentioned in the Introduction, one particular topological quality we are interested in is the relative fractal dimension of the network, which we will implement using a Mass-Excluded Box Burning algorithm, or a Compact Box Burning algorithm, both from \cite{song_how_2007}. 
	
	In addition to changing the parameters that govern attachment and information flow, we can also control the dynamics of the resource node. Possible implementations are listed below:
	\begin{enumerate}
		\item The simplest instantiation would be to begin with a finite number of resources and simply let the network grow until all the resources have been extracted, and the network is allowed to whither and die. In this case, arguably the most interesting questions will be how long it takes the network to die after resources are exhausted, and how the collapse unfolds.
		\item A variation on this would be to make resource extraction from the central node harder in proportion to how much has been exhausted (this would be analogous to the current fossil fuel situation, where, as a greater percentage of the oil fields are exploited, further exploitation because more resource intensive). Here we might be able to observe a "peak oil"-like effect and a slower collapse.
		\item Alternately, we could allow the resource node's value to remain constant (as if it were a renewable resource). Would the network grow on for infinity, or eventually reach a steady-state, where it was large enough that all usable energy was dissipated in the network and further growth (but no die-off) was achieved. 
		\item Finally, we could attempt to model the transition from a fossil-fuel based system to a renewable-based system, where the resource node is originally instantiated as finite, but, at some point, makes a transition to a much smaller, but replenishing state. How does the network adapt to this large perturbation? Does it smoothly transition, or is there a significant die-off, followed by re-growth?
	\end{enumerate}
	
	All of the methods described here attempt to describe the network as a whole entity. The behaviour of the individual nodes is of less relevance, as they have been pre-programmed with certain behaviours. 
	
	\bibliographystyle{unsrt}
	\bibliography{Dissipative_Structures.bib}

\end{document}